\section{Sequential Quadratic Programming (SQP)}

% ------------------------------------------------
\begin{frame}{What is SQP?}
\textbf{Idea:} Solve a nonlinear, constrained problem by repeatedly solving a \emph{quadratic program (QP)} built from local models.\\[4pt]
\begin{itemize}
  \item Linearize constraints; quadratic model of the Lagrangian/objective.
  \item Each iteration: solve a QP to get a step \(d\), update \(x \leftarrow x + \alpha d\).
  \item Strength: strong local convergence (often superlinear) with good Hessian info.
\end{itemize}
\end{frame}

% ------------------------------------------------
\begin{frame}{Target Problem (NLP)}
\[
\min_{x \in \R^n} \ f(x)
\quad
\text{s.t.}\quad
g(x)=0,\quad h(x)\le 0
\]
\begin{itemize}
  \item \(f:\R^n\!\to\!\R\), \(g:\R^n\!\to\!\R^{m}\) (equalities), \(h:\R^n\!\to\!\R^{p}\) (inequalities).
  \item KKT recap (at candidate optimum \(x^\star\)): 
\[
\exists \ \lambda \in \R^{m},\ \mu \in \R^{p}_{\ge 0}:
\ \grad f(x^\star) + \nabla g(x^\star)^T\lambda + \nabla h(x^\star)^T \mu = 0,
\]
\[
g(x^\star)=0,\quad h(x^\star)\le 0,\quad \mu \ge 0,\quad \mu \odot h(x^\star) = 0.
\]
\end{itemize}
\end{frame}

% ------------------------------------------------
\begin{frame}{From NLP to a QP (Local Model)}
At iterate \(x_k\) with multipliers \((\lambda_k,\mu_k)\):\\[4pt]
\textbf{Quadratic model of the Lagrangian}
\[
m_k(d) = \ip{\grad f(x_k)}{d} + \tfrac{1}{2} d^T B_k d
\]
with \(B_k \approx \nabla^2_{xx}\Lag(x_k,\lambda_k,\mu_k)\).\\[6pt]
\textbf{Linearized constraints}
\[
g(x_k) + \nabla g(x_k)\, d = 0,\qquad
h(x_k) + \nabla h(x_k)\, d \le 0.
\]
\end{frame}

% ------------------------------------------------
\begin{frame}{The SQP Subproblem (QP)}
\[
\begin{aligned}
\min_{d \in \R^n}\quad & \grad f(x_k)^T d + \tfrac{1}{2} d^T B_k d \\
\text{s.t.}\quad & \nabla g(x_k)\, d + g(x_k) = 0, \\
& \nabla h(x_k)\, d + h(x_k) \le 0.
\end{aligned}
\]
\begin{itemize}
  \item Solve QP \(\Rightarrow\) step \(d_k\) and updated multipliers \((\lambda_{k+1},\mu_{k+1})\).
  \item Update \(x_{k+1} = x_k + \alpha_k d_k\) (line search or trust-region).
\end{itemize}
\end{frame}

% ------------------------------------------------
\begin{frame}{Algorithm Sketch (SQP)}
\begin{enumerate}
  \item Start with \(x_0\), multipliers \((\lambda_0,\mu_0)\), and \(B_0 \succ 0\).
  \item Build QP at \(x_k\) with \(B_k\), linearized constraints.
  \item Solve QP \(\Rightarrow\) get \(d_k\), \((\lambda_{k+1},\mu_{k+1})\).
  \item Globalize: line search on merit or use filter/TR to choose \(\alpha_k\).
  \item Update \(x_{k+1} = x_k + \alpha_k d_k\), update \(B_{k+1}\) (e.g., BFGS).
\end{enumerate}
\end{frame}

% ------------------------------------------------
\begin{frame}{Toy Example (Local Models)}
\textbf{Problem:}
\[
\min_{x\in\R^2} \ \tfrac{1}{2}\norm{x}^2
\quad \text{s.t.} \quad g(x)=x_1^2 + x_2 - 1 = 0,\ \ h(x)=x_2 - 0.2 \le 0.
\]
At \(x_k\), build QP with
\[
\grad f(x_k)=x_k,\quad B_k=I,\quad
\nabla g(x_k) = \begin{bmatrix} 2x_{k,1} & 1 \end{bmatrix},\ 
\nabla h(x_k) = \begin{bmatrix} 0 & 1 \end{bmatrix}.
\]
Solve for \(d_k\), then \(x_{k+1}=x_k+\alpha_k d_k\).
\end{frame}
 

% ------------------------------------------------
\begin{frame}{Globalization: Making SQP Robust}
SQP is an important method, and there are many issues to be considered to obtain an \textbf{efficient} and \textbf{reliable} implementation:
\begin{itemize}
  \item Efficient solution of the linear systems at each Newton Iteration (Matrix block structure can be exploited.
  \item Quasi-Newton approximations to the Hessian.
  \item Trust region, line search, etc. to improve robustnes (i.e TR: restrict \(\norm{d}\) to maintain model validity.)
  \item Treatment of constraints (equality and inequality) during the iterative process.
  \item Selection of good starting guess for $\lambda$.
\end{itemize}
\end{frame}

 
   
 
 

% ------------------------------------------------
\begin{frame}{Final Takeaways on SQP}
\textbf{When SQP vs.\ Interior-Point?}
\begin{itemize}
  \item \textbf{SQP}: strong local convergence; warm-start friendly; natural for NMPC.
  \item \textbf{IPM}: very robust for large, strictly feasible problems; good for dense inequality sets.
  \item In practice: both are valuable—choose to match problem structure and runtime needs.
\end{itemize} 
\textbf{Takeaways of SQP} 
\begin{itemize}
  \item SQP = Newton-like method using a sequence of structured QPs.
  \item Globalization (merit/filter/TR) makes it reliable from poor starts.
  \item Excellent fit for control (NMPC/trajectory optimization) due to sparsity and warm starts.
\end{itemize}
\end{frame}
